\hsize=6.25truein
\vsize=9.65truein
\tolerance=15000
\pretolerance=15000
\parindent=0.0in
\font\sym=cmex10
\def\up#1{\raise3em\hbox{#1}}
\def\textindent#1{\noindent\llap{#1\enspace}\ignorespaces}
\def\llap#1{\noindent\hbox to \parindent{#1\hfil}}
\def\uncatcodespecials{\def\do##1{\catcode`##1=12 }\dospecials}
\def\setupverbatim{\tt
 \obeylines \uncatcodespecials \obeyspaces }
{\obeyspaces\global\let =\ } %
\def\verbatim{\begingroup\setupverbatim\doverbatim}
\def\doverbatim#1{\def\next##1#1{##1\endgroup}\next}
\font\dunh=cmdunh10 at 10 truept
\font\vtt=cmvtt10 at 10 truept
\font\ssdc=cmssdc10 at 10 truept
\font\sans=cmss10 at 10 truept
\font\fib=cmfib8 at 8 truept
\font\fun=cmff10 at 10 truept
\font\ttt=cmtt10 at 10 truept
\def\today{\ifcase\month\or
   January\or February\or March\or April\or May\or June\or July\or
   August\or September\or October\or November\or December\fi
   \space\number\year}
\def\bks{$\backslash$}
\def\display{\obeylines\parindent=0.5in\ttt}
\def\dv1{\bigskip\bigskip}
\def\dvd{\bigskip\bigskip\filbreak}
\font\largeb=cmbx10 scaled\magstep1
\font\larger=cmr10 scaled\magstep1
\font\bigbig=cmr10 scaled\magstep4
\font\ft=cmr5 at 5 truept
\font\st=cmr7 at 7 truept
\footline{\rm MAP\hfil\folio\hfil\today}
Reference: MAP\_NEURAL\_AC1TEMP
\medskip
\centerline{\bf MAP PROGRAM LIBRARY}
\bigskip
\bigskip
\bigskip
{\largeb  Subroutine MAP\_NEURAL\_AC1TEMP}
\bigskip\bigskip
\parindent=0.5in
\item {\bf 1.} {\largeb Provenance of Source Code}
\bigskip
\item {} H.K.D.H. Bhadeshia, Phase Transformations Group, Department of Materials Science and Metallurgy, University of Cambridge, Cambridge, UK.
\dv1
\item {\bf 2.} {\largeb  Purpose}
\bigskip
\item {} To predict the $Ac_1$ temperature of steel as a function of the chemical 
composition and heating rate.
\dv1
\item{\bf 3.} {\largeb  Specification}
\bigskip
\item {} ~SUBROUTINE MAP\_NEURAL\_AC1TEMP(AW,W1,W2,THETA1,THETA2,AMIN,AMAX,
\item {} \&IMAX,Y,HT,IERR,IN,IHID,ERROR)
\medskip
\item {} ~DOUBLE PRECISION AW(22),W1(4,22),W2(4),THETA1(4),THETA2,
\item {} \&AMIN(22),AMAX(22),Y(20),HT(20),ERROR
\medskip
\item {} ~INTEGER IERR(20),IMAX,IN,IHID
\dv1
\item{\bf 4.} {\largeb  Description}
\bigskip
\item {} MAP\_NEURAL\_AC1TEMP uses neural network analysis to predict the
temperature of the onset of austenite formation.
\medskip
\item {} The neural net was trained using 394 of a database of 788 examples
constructed using information from Refs [2 - 7]. The remaining 394 examples
were used as `new' experiments to test the trained network. This database
is in the file ``neural\_dataset''.
\medskip
\item {} Two input data files are needed for programs using MAP\_NEURAL\_AC1TEMP -
\item {} ACINPUT (which contains the information on the chemical composition of the
steel, and is read into the array AW), and AC1 (which contains neural information,
and should not be altered).
\dv1
\item{\bf 5.} {\largeb  References}
\bigskip
\item {} 1. L. Gavard, H.K.D.H. Bhadeshia, D.J.C. MacKay, and S. Suzuki,
 Bayesian Neural Network Model for Austenite Formation in Steels,
{\it Materials Science and Technology}, 1996, $\underline {12}$, 453-463.
\medskip
\item {} 2. G.F. Vander Vroot, ed., {\it Atlas of 
Time-Temperature-Transformation Diagrams for Irons and Steels}, ASM 
International, Ohio, USA, (1991).
\medskip
\item {} 3. Special Report 56, {\it Atlas of Isothermal Transformation
Diagrams of B.S. En Steels}, 2nd edition, Iron and Steel Institute, London,
(1956).
\medskip
\item {} 4. T. Cool, Systematic Design of Welding Alloys for Power Plant
Steels, CPGS Thesis, University of Cambridge, (1994).
\medskip
\item {} 5. K. Akibo, {\it Scientific American (Japanese Edition)},
(January 1993), 20-29.
\medskip
\item {} 6. R. Reed, Ph.D. Thesis, University of Cambridge, (1987).
\medskip
\item {} 7. Phase Transformation Kinetics and Hardenability of Medium Alloy
Steels, Climax \item {} Molybdenum Company, Connecticut, USA, (1972).
\dv1
\item{\bf 6.} {\largeb  Parameters}
\bigskip
\item {} {\bf Input parameters}
\bigskip
\item {} AW - real array of dimension 22
\item {} \indent AW contains the input data from the file ACINPUT, which
includes the chemical \indent composition of the steel. See Section 9.2 for details
of an example file.
\medskip
\item {} AMIN - real array of dimension 22
\item {} \indent AMIN contains the minimum unnormalised values of the input
variables (from the \indent input file AC1).
\medskip
\item {} AMAX - real array of dimension 22
\item {} \indent AMAX contains the maximum unnormalised values of the input 
variables (from the \indent input file AC1).
\medskip
\item {} THETA1 - real array of dimension 4
\item {} \indent THETA1 contains the biases associated with W1, and is
read in from the file AC1.
\medskip
\item {} THETA2 - real 
\item {} \indent THETA2 is the bias associated with W2, and is read
in from the file AC1.
\medskip
\item {} W1 - real array of dimension (4,22)
\item {} \indent W1 contains weights read in from the file AC1, which are
coefficients used in the \indent prediction of the $Ac_1$ temperature.
\medskip
\item {} W2 - real array of dimension 4
\item {} \indent W2 contains weights read in from the file AC1, which are
coefficients used in the \indent  prediction of the $Ac_1$ temperature.
\medskip
\item {} IN - integer
\item {} \indent IN is the number of inputs (22 for this model).
\medskip
\item {} IHID - integer
\item {} \indent IHID is the number of hidden units (4 for this model).
\medskip
\item {} IMAX - integer
\item {} \indent IMAX is the number of heating rates calculated ($<$ 20).
\dv1
\item {} {\bf Output parameters}
\bigskip
\item {} HT - real array of dimension IMAX ($<$ 20)
\item {} \indent HT contains the heating rates (in kelvin per second).
\medskip
\item {} Y - real array of dimension IMAX
\item {} \indent Y contains the $Ac_1$ temperature for each HT.
\medskip
\item {} IERR - integer array of dimension IMAX
\item {} \indent IERR=0 if all inputs are within the range of the training
dataset for the neural network \indent (so the 95\% confidence error bars are at about
$\pm$ 11\%). 
\medskip
\item {} \indent IERR=1 if some inputs are outside the range of the training 
set for the neural network \indent (so the 95\% confidence error bars may then be greater than
$\pm$ 11\%).
\medskip
\item {} \indent See Ref [1] for further details.
\medskip
\item {} ERROR - real
\item {} \indent ERROR is set at $\pm$ ERROR/2 for 95\% confidence error limits (in joules).
\dv1
\vfill \eject
\item{\bf 7.} {\largeb  Error Indicators}
\bigskip
\item {} IERR is an integer array of dimension IMAX. 
\medskip
\item {} \indent IERR=0 if all inputs are within the range of the training
dataset for the neural network \indent (so the 95\% confidence error bars are at about
$\pm$ 11\%).
\medskip
\item {} \indent IERR=1 if some inputs are outside the range of the training
set for the neural network \indent (so the 95\% confidence error bars may then be greater than
$\pm$ 11\%).
\dv1
\item{\bf 8.} {\largeb  Accuracy}
\medskip
\item {} N/A
\dv1
\item{\bf 9.} {\largeb  Further Comments}
\bigskip
\item {} Used in the program MAP\_NEURAL\_STEEL.
\dv1
\item{\bf 10.} {\largeb  Example}
\bigskip
\item{} {\bf 10.1 Program text}
\bigskip
\item {}       ~DOUBLE PRECISION W1(4,22), W2(4),THETA1(4), THETA2
\item {}    \&AMIN(23),AMAX(23),AW(22),Y(20),HT(20),ERROR
\item {}       ~INTEGER IN,IHID,IDUM,IMAX,IERR(20)
\item {}        ~IN=22
\item {}        ~IHID=4
\item {}        ~IMAX=5
\item {}        ~OPEN(UNIT=1, FILE='ACINPUT')
\item {}        ~DO 10 I=1,22
\item {}        ~~~~READ(1,*) AW(I)
\item {~ 10}    ~CONTINUE
\item {}        ~OPEN(UNIT=2, FILE='AC1')
\item {}        ~DO 20 I=1,IN+1
\item {}        ~~~~READ(2,*) IDUM,AMIN(I),AMAX(I)
\item {~ 20}    ~CONTINUE
\item {}       ~DO 30 I=1,IHID
\item {}       ~~~~READ(2,*) THETA1(I)
\item {}       ~~~~DO 40 J=1,IN
\item {}       ~~~~~~~READ(2,*) W1(I,J)
\item {~ 40}   ~~~~CONTINUE
\item {~ 30}   ~CONTINUE
\item {}       ~READ(2,*) THETA2
\item {}       ~DO 50 I=1,IHID
\item {}       ~~~~READ(2,*) W2(I)
\item {~ 50}   ~CONTINUE
\item {}       ~CALL AC1TEMP(AW,W1,W2,THETA1,THETA2,AMIN,AMAX,IMAX,Y,HT,
\item {}       \&IERR,IN,IHID,ERROR)
\item {}       ~DO 100 I=1,IMAX
\item {}       ~~~~WRITE(6,200) HT(I),Y(I),IERR(I)
\item {~ 200}  ~~~~FORMAT(5X,2F11.4,2X,I2)
\item {~ 100}  ~CONTINUE
\item {}        ~STOP
\item {}        ~END
\dv1
\vfill \eject
\item {} {\bf 10.2 Program data}
\bigskip
\item {} An example ACINPUT file:-
\bigskip
\item {0.2}\indent	 C
\item {0.0}\indent 	 Si
\item {0.0}\indent	 Mn
\item {0.000}\indent 	 S
\item {0.000}\indent	 P
\item {0.00}\indent	 Cu
\item {0.0}\indent	Ni
\item {5.0}\indent	Cr
\item {0.0}\indent	Mo
\item {0.0}\indent	Nb
\item {0.0}\indent	V
\item {0.00}\indent	Ti
\item {0.00}\indent     Al
\item {0.000}\indent   B
\item {0.0}\indent     W
\item {0.000}\indent	As
\item {0.000}\indent	Sn
\item {0.000}\indent	Zr
\item {0.0}\indent	Co
\item {0.000}\indent	N
\item {0.000}\indent	O
\item {1.0}\indent	Heating rate
\dv1
\item {} {\bf 10.3 Program results}
\bigskip
\item {0.0100} \indent  855.7674 \indent  1
\item {0.1000} \indent  855.9666 \indent  0
\item {1.0000} \indent  857.9337 \indent  0
\item {10.000} \indent  874.2388 \indent  0
\item {100.00} \indent  636.1600 \indent  1
\dv1
\item{\bf 11.} {\largeb  Auxiliary Routines}
\bigskip
\item {} None
\dv1
\item{\bf 12.} {\largeb  Keywords}
\bigskip
\item {} neural net, austenite formation
\bye

